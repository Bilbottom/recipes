\documentclass[12pt]{article}%[a4paper,10pt]

% A Few Useful Packages
\usepackage{marvosym,fontspec,titlesec,charter}
\usepackage{xunicode,xltxtra,url,parskip}
\RequirePackage{color,graphicx}
\usepackage[usenames,dvipsnames]{xcolor}
\usepackage[margin=0.8in]{geometry}

% Setup hyperref package, and colours for links
\usepackage{hyperref}
\definecolor{linkcolour}{rgb}{0,0.2,0.6}
\hypersetup{colorlinks,breaklinks,urlcolor=linkcolour,linkcolor=linkcolour}

\usepackage{fancyhdr}\pagestyle{fancy}
\renewcommand{\headrulewidth}{0pt}% Default header rule
\renewcommand{\footrulewidth}{0pt}% No footer rule
\fancyhf{}% Clear header/footer
\fancypagestyle{plain}{
  \renewcommand{\headrulewidth}{0pt}% No header rule
  \renewcommand{\footrulewidth}{0pt}% Default footer rule
  \fancyhf{}% Clear header/footer
}
\AtBeginDocument{\thispagestyle{plain}}

\setlength{\parindent}{0pt}
\setlength{\parskip}{.5\baselineskip plus 1pt minus 1pt}

\setlength{\headheight}{0pt}

%%%%%%%%%%%%%%%%%%%%%%%%%%%%%%%%%%%%%%%%%%%%%%%%%%%%%

% FONTS
\defaultfontfeatures{Mapping=tex-text}
\setmainfont[
    Path = ../fonts/,
    SmallCapsFont = Fontin-SmallCaps.otf,
    BoldFont = Fontin-Bold.otf,
    ItalicFont = Fontin-Italic.otf
]
% A font by Jos Buivenga (exljbris) -> www.exljbris.com
{Fontin.otf}
%%%

%%%%%%%%%%
% Colors %
%%%%%%%%%%

\RequirePackage{xcolor}

\definecolor{white}{RGB}{255,255,255}
\definecolor{gray}{HTML}{4D4D4D}
\definecolor{lightgray}{HTML}{D3D3D3}
\definecolor{darkgrey}{HTML}{404040}


\RequirePackage{enumitem,wasysym} %  https://tex.stackexchange.com/questions/85335/how-to-change-dot-spacing-in-dotfill
\newcommand{\tick}{\dotfill\(\ocircle\)}
\newcommand{\blue}[1]{\textcolor{blue!70!white}{#1}}
\newcommand{\red}[1]{\textcolor{red}{#1}}
\newcommand{\green}[1]{\textcolor{green!70!black}{#1}}
\newcommand{\quant}[1]{\red{#1}\tick}

%%%%%%%%%%%%%%%%%%%%%%%%%%%%%%%%%%%%%%%%%%%%%%%%%%%%%%%%%%%%%%%%%%%%%%%%%%%%%%%%%%%%%%%%%%%%%%%%%%%%
%%%%%%%%%%%%%%%%%%%%%%%%%%%%%%%%%%%%%%%%%%%%%%%%%%%%%%%%%%%%%%%%%%%%%%%%%%%%%%%%%%%%%%%%%%%%%%%%%%%%

%--------------------BEGIN DOCUMENT----------------------
\begin{document}

% \vspace*{\baselineskip}

{\LARGE \blue{Sweet Potato \& Coconut Soup}\hfill\normalsize 2-3 people}

\vspace{-\baselineskip}\hrulefill

% \vspace{-0.5\baselineskip}\hfill {\footnotesize Prep \green{25 minutes}, Cook \green{2 hours}}

\subsection*{Ingredients}

\begin{itemize}[nolistsep]
    \item Butter --- \quant{50g}
    \item Red Chilli Pepper --- \quant{1, finely chopped}
    \item Ginger --- \quant{1 teaspoon, grated}
    \item Garlic --- \quant{3 cloves, crushed}
    \item Carrots --- \quant{125g, chopped small}
    \item Leeks --- \quant{125g, chopped small}
    \item Celery --- \quant{125g, chopped small}
    \item Onion --- \quant{125g, chopped small}
    \item Plain Flour --- \quant{50g}
    \item Chicken Stock --- \quant{2 pints}
    \item Sweet Potatoes --- \quant{500g}
    \item Coconut Milk --- \quant{1 tin}
    \item Fresh Coriander --- \quant{1 tablespoon}
    \item Soy Sauce --- \quant{1 teaspoon}
    \item Spring Onions --- \quant{3, finely chopped (optional)}
\end{itemize}

\subsection*{Instructions}

\begin{enumerate}
    \item Melt the \blue{butter} in a large pot.
    \item Add the \blue{red chilli pepper}, \blue{ginger}, \blue{garlic}, \blue{carrots}, \blue{leeks}, \blue{celery}, and \blue{onion}. Stir for about 15 minutes until softened.
    \item Mix in the \blue{flour} to make the mixture thick and stodgy.
    \item Pour in the \blue{chicken stock} gradually.
    \item Add the \blue{sweet potatoes}, bringing the water to the boil. Cover and simmer for 20 minutes.
    \item Blend with an electric hand blender for 30-60 seconds.
    \item Add the \blue{coconut milk}, \blue{coriander}, \blue{soy sauce}, and \blue{spring onions}. Serve immediately.
\end{enumerate}

\vspace{\baselineskip} This can be made in advance, but must be reheated slowly (like all soups).

\end{document}
